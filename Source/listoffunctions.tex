\section {Список по идентификатору}

Вначале идут функции многомерной оптимизации, потом двумерной, далее одномерной. В конце идут функции бинарной оптимизации.

\begin{enumerate}
 \item \hyperref[TestFunctions:section:HML_TestFunction_Ackley]{\textbf{HML\_TestFunction\_Ackley}} --- функция Ackley;
 \item \hyperref[TestFunctions:section:HML_TestFunction_GaussianQuartic]{\textbf{HML\_TestFunction\_GaussianQuartic}} --- функция Gaussian quartic;
 \item \hyperref[TestFunctions:section:HML_TestFunction_Griewangk]{\textbf{HML\_TestFunction\_Griewangk}} --- функция Гриванка;
 \item \hyperref[TestFunctions:section:HML_TestFunction_HyperEllipsoid]{\textbf{HML\_TestFunction\_HyperEllipsoid}} --- гипер-эллипсоид;
 \item \hyperref[TestFunctions:section:HML_TestFunction_ParaboloidOfRevolution]{\textbf{HML\_TestFunction\_ParaboloidOfRevolution}} --- эллиптический параболоид;
 \item \hyperref[TestFunctions:section:HML_TestFunction_Rastrigin]{\textbf{HML\_TestFunction\_Rastrigin}} --- функция Растригина;
 \item \hyperref[TestFunctions:section:HML_TestFunction_RastriginNovgorod]{\textbf{HML\_TestFunction\_RastriginNovgorod}} --- функция Растригина новгородская;
 \item \hyperref[TestFunctions:section:HML_TestFunction_Rosenbrock]{\textbf{HML\_TestFunction\_Rosenbrock}} --- функция Розенброка;
 \item \hyperref[TestFunctions:section:HML_TestFunction_RotatedHyperEllipsoid]{\textbf{HML\_TestFunction\_RotatedHyperEllipsoid}} --- развернутый гипер-эллипсоид;
 \item \hyperref[TestFunctions:section:HML_TestFunction_Schwefel]{\textbf{HML\_TestFunction\_Schwefel}} --- функция Швефеля;
 \item \hyperref[TestFunctions:section:HML_TestFunction_StepFunction]{\textbf{HML\_TestFunction\_StepFunction}} --- функция Step (модифицированная версия De Jong 3);
 \item \hyperref[TestFunctions:section:HML_TestFunction_AdditivePotential]{\textbf{HML\_TestFunction\_AdditivePotential}} --- аддитивная потенциальная функция;
 \item \hyperref[TestFunctions:section:HML_TestFunction_Bosom]{\textbf{HML\_TestFunction\_Bosom}} --- функция Bosom;
 \item \hyperref[TestFunctions:section:HML_TestFunction_EggHolder]{\textbf{HML\_TestFunction\_EggHolder}} --- функция Egg Holder;
 \item \hyperref[TestFunctions:section:HML_TestFunction_Himmelblau]{\textbf{HML\_TestFunction\_Himmelblau}} --- функция Химмельблау;
 \item \hyperref[TestFunctions:section:HML_TestFunction_InvertedRosenbrock]{\textbf{HML\_TestFunction\_InvertedRosenbrock}} --- перевернутая функция Розенброка;
 \item \hyperref[TestFunctions:section:HML_TestFunction_Katnikov]{\textbf{HML\_TestFunction\_Katnikov}} --- функция Катникова;
 \item \hyperref[TestFunctions:section:HML_TestFunction_Multiextremal3]{\textbf{HML\_TestFunction\_Multiextremal3}} --- функция Multiextremal3;
 \item \hyperref[TestFunctions:section:HML_TestFunction_Multiextremal4]{\textbf{HML\_TestFunction\_Multiextremal4}} --- функция Multiextremal4;
 \item \hyperref[TestFunctions:section:HML_TestFunction_MultiplicativePotential]{\textbf{HML\_TestFunction\_MultiplicativePotential}} --- мультипликативная потенциальная функция;
 \item \hyperref[TestFunctions:section:HML_TestFunction_Rana]{\textbf{HML\_TestFunction\_Rana}} --- функция Rana;
 \item \hyperref[TestFunctions:section:HML_TestFunction_RastriginWithChange]{\textbf{HML\_TestFunction\_RastriginWithChange}} --- функция Растригина с изменением коэффициентов;
 \item \hyperref[TestFunctions:section:HML_TestFunction_RastriginWithTurning]{\textbf{HML\_TestFunction\_RastriginWithTurning}} --- функция Растригина овражная с поворотом осей;
 \item \hyperref[TestFunctions:section:HML_TestFunction_ReverseGriewank]{\textbf{HML\_TestFunction\_ReverseGriewank}} --- функция ReverseGriewank;
 \item \hyperref[TestFunctions:section:HML_TestFunction_ShekelsFoxholes]{\textbf{HML\_TestFunction\_ShekelsFoxholes}} --- функция "Лисьи норы" Шекеля;
 \item \hyperref[TestFunctions:section:HML_TestFunction_Sombrero]{\textbf{HML\_TestFunction\_Sombrero}} --- функция Сомбреро;
 \item \hyperref[TestFunctions:section:HML_TestFunction_Multiextremal]{\textbf{HML\_TestFunction\_Multiextremal}} --- функция Multiextremal;
 \item \hyperref[TestFunctions:section:HML_TestFunction_Multiextremal2]{\textbf{HML\_TestFunction\_Multiextremal2}} --- функция Multiextremal2;
 \item \hyperref[TestFunctions:section:HML_TestFunction_Wave]{\textbf{HML\_TestFunction\_Wave}} --- волна;
 \item \hyperref[TestFunctions:section:HML_TestFunction_SumVector]{\textbf{HML\_TestFunction\_SumVector}} --- сумма всех элементов бинарного вектора.
\end{enumerate}

\section {Список формул}

\begin{enumerate}
 \item \hyperref[TestFunctions:section:HML_TestFunction_Ackley]{\textbf{HML\_TestFunction\_Ackley}} --- функция Ackley:
 \begin{equation}
 f\left( \bar{x}\right) = 20 + e - 20e^{-0.2\sqrt{\frac{1}{n}\sum_{i=1}^{n}\bar{x}_i^2}}-e^{\frac{1}{n}\sum_{i=1}^{n}cos\left( 2\pi\cdot\bar{x}_i\right) }.
 \end{equation}
 \item \hyperref[TestFunctions:section:HML_TestFunction_GaussianQuartic]{\textbf{HML\_TestFunction\_GaussianQuartic}} --- функция Gaussian quartic:
 \begin{equation}
 f\left( \bar{x}\right) = \sum_{i=1}^{n}\left( i\cdot\bar{x}_i^4\right) +rnorm\left( 0,1\right).
 \end{equation}
 \item \hyperref[TestFunctions:section:HML_TestFunction_Griewangk]{\textbf{HML\_TestFunction\_Griewangk}} --- функция Гриванка:
\begin{equation}
f\left( \bar{x}\right) = \sum_{i=1}^{n}\dfrac{\bar{x}_i^2}{4000}-\prod_{i=1}^{n}\cos\left( \dfrac{\bar{x}_i}{\sqrt{i}}\right)+1.
\end{equation} \item \hyperref[TestFunctions:section:HML_TestFunction_HyperEllipsoid]{\textbf{HML\_TestFunction\_HyperEllipsoid}} --- гипер-эллипсоид:
\begin{equation}
f\left( \bar{x}\right) = \sum_{i=1}^{n}\left( i\cdot\bar{x}_i\right) ^2.
\end{equation}
 \item \hyperref[TestFunctions:section:HML_TestFunction_ParaboloidOfRevolution]{\textbf{HML\_TestFunction\_ParaboloidOfRevolution}} --- эллиптический параболоид:
 \begin{equation}
 f\left( \bar{x}\right) = \sum_{i=1}^{n}\bar{x}_i^2.
 \end{equation}
 \item \hyperref[TestFunctions:section:HML_TestFunction_Rastrigin]{\textbf{HML\_TestFunction\_Rastrigin}} --- функция Растригина:
 \begin{equation}
 f\left( \bar{x}\right) = 10n+\sum_{i=1}^{n}\left( \bar{x}_i^2-10\cdot\cos\left( 2\pi\cdot \bar{x}_i\right) \right).
 \end{equation}
 \item \hyperref[TestFunctions:section:HML_TestFunction_RastriginNovgorod]{\textbf{HML\_TestFunction\_RastriginNovgorod}} --- функция Растригина новгородская:
 \begin{equation}
 f\left( \bar{x}\right) = n+\sum_{i=1}^{n}\left( \bar{x}_i^2-\cos\left(18\cdot \bar{x}_i^2\right) \right).
 \end{equation}
 \item \hyperref[TestFunctions:section:HML_TestFunction_Rosenbrock]{\textbf{HML\_TestFunction\_Rosenbrock}} --- функция Розенброка:
 \begin{equation}
 f\left( \bar{x}\right) = \sum_{i=1}^{n-1} \left( 100{\left( \bar{x}_{i+1}-\bar{x}_i^2\right)}^2+{\left( 1-\bar{x}_i\right) }^2 \right).
 \end{equation}
 \item \hyperref[TestFunctions:section:HML_TestFunction_RotatedHyperEllipsoid]{\textbf{HML\_TestFunction\_RotatedHyperEllipsoid}} --- развернутый гипер-эллипсоид:
 \begin{equation}
 f\left( \bar{x}\right) = \sum_{i=1}^{n}\left( \sum_{j=1}^{j}\bar{x}_j\right) ^2.
 \end{equation}
 \item \hyperref[TestFunctions:section:HML_TestFunction_Schwefel]{\textbf{HML\_TestFunction\_Schwefel}} --- функция Швефеля:
 \begin{equation}
 f\left( \bar{x}\right) = 418.9829 n-\sum_{i=1}^{n}\left( \bar{x}_i\sin\left( \sqrt{\left| \bar{x}_i\right|}\right)  \right).
 \end{equation}
 \item \hyperref[TestFunctions:section:HML_TestFunction_StepFunction]{\textbf{HML\_TestFunction\_StepFunction}} --- функция Step (модифицированная версия De Jong 3):
 \begin{equation}
 f\left( \bar{x}\right) =\left\lbrace \begin{aligned}
 \sum_{i=1}^{n} \left( int\left( \bar{x}_{i}\right)  \right)^2,& \text{ если} \sum_{i=1}^{n} \left| int\left( \bar{x}_{i}\right)\right| \neq 0 ;\\ \left( \sum_{i=1}^{n} \left| \bar{x}_{i}\right|\right) -1 ,& \text{ иначе}.
 \end{aligned}\right.
 \end{equation}
 \item \hyperref[TestFunctions:section:HML_TestFunction_AdditivePotential]{\textbf{HML\_TestFunction\_AdditivePotential}} --- аддитивная потенциальная функция:
 \begin{equation}
 f\left( \bar{x}\right) = z\left( \bar{x}_1\right)+ z\left( \bar{x}_2\right), \text{ где}
 \end{equation}
 \begin{equation*}
 z\left( v\right)= -\dfrac{1}{\left( v-1\right)^2+0.2 }-\dfrac{1}{2\left( v-2\right)^2+0.15}-\dfrac{1}{3\left( v-3\right)^2+0.3}.
 \end{equation*}
\item \hyperref[TestFunctions:section:HML_TestFunction_Bosom]{\textbf{HML\_TestFunction\_Bosom}} --- функция Bosom:
\begin{equation}
f\left( \bar{x}\right) = e^{\frac{z_1\left( \bar{x}_1,\bar{x}_2\right)}{1000} } +e^{\frac{z_2\left( \bar{x}_1,\bar{x}_2\right)}{1000} }+ 0.15\cdot e^{z_1\left( \bar{x}_1,\bar{x}_2\right) }+0.15\cdot e^{z_2\left( \bar{x}_1,\bar{x}_2\right) }, \text{ где}
\end{equation}
\begin{equation*}
\label{TestFunctions:eq:HML_Bosom2}
z_1\left( \bar{x}_1,\bar{x}_2\right) = -\left(\left( \bar{x}_1+4\right)^2 +\left( \bar{x}_2+4\right)^2 \right)^2 ,
\end{equation*}
\begin{equation*}
\label{TestFunctions:eq:HML_Bosom3}
z_2\left( \bar{x}_1,\bar{x}_2\right) = -\left(\left( \bar{x}_1-4\right)^2 +\left( \bar{x}_2-4\right)^2 \right)^2 ,
\end{equation*}
 \item \hyperref[TestFunctions:section:HML_TestFunction_EggHolder]{\textbf{HML\_TestFunction\_EggHolder}} --- функция Egg Holder:
 \begin{equation}
 f\left( \bar{x}\right) = -\bar{x}_1\sin\left( \sqrt{\left| \bar{x}_1-47-\bar{x}_2\right| }\right)- (\bar{x}_2+47)\sin\left( \sqrt{\left| \dfrac{\bar{x}_1}{2}+47+\bar{x}_2\right| }\right).
 \end{equation}
 \item \hyperref[TestFunctions:section:HML_TestFunction_Himmelblau]{\textbf{HML\_TestFunction\_Himmelblau}} --- функция Химмельблау:
 \begin{equation}
 f\left( \bar{x}\right) = \left( \bar{x}_1^2+\bar{x}_2-11\right)^2+\left( \bar{x}_1+\bar{x}^2-7\right)^2.
 \end{equation}
 \item \hyperref[TestFunctions:section:HML_TestFunction_InvertedRosenbrock]{\textbf{HML\_TestFunction\_InvertedRosenbrock}} --- перевернутая функция Розенброка:
 \begin{equation}
 f\left( \bar{x}\right) =\dfrac{-100}{100\left( \bar{x}_1^2-\bar{x}_2\right) +\left( 1.-\bar{x}_1\right)^2+600}.
 \end{equation}
 \item \hyperref[TestFunctions:section:HML_TestFunction_Katnikov]{\textbf{HML\_TestFunction\_Katnikov}} --- функция Катникова:
 \begin{equation}
 f\left( \bar{x}\right) = 0.5\left( \bar{x}_1^2+\bar{x}_2^2\right) \left( 2A+A\cos\left( 1.5\bar{x}_1\right)\cos\left( 3.14\bar{x}_2\right)+A\cos\left( \sqrt{5}\bar{x}_1\right)\cos\left( 3.5\bar{x}_2\right)    \right).
 \end{equation}
 \item \hyperref[TestFunctions:section:HML_TestFunction_Multiextremal3]{\textbf{HML\_TestFunction\_Multiextremal3}} --- функция Multiextremal3:
 \begin{equation}
 f\left( \bar{x}\right) = \bar{x}_1^2\left| \sin\left( 2\bar{x}_1\right) \right| +\bar{x}_2^2\left| \sin\left( 2\bar{x}_2\right) \right| -\dfrac{1}{5\bar{x}_1^2+5\bar{x}_2^2+0.2} + 5.
 \end{equation}
 \item \hyperref[TestFunctions:section:HML_TestFunction_Multiextremal4]{\textbf{HML\_TestFunction\_Multiextremal4}} --- функция Multiextremal4:
 \begin{align}
 f\left( \bar{x}\right) =& 0.5\left( \bar{x}_1^2+\bar{x}_1\bar{x}_2 +\bar{x}_2^2\right) \left( 1+0.5\cos\left(1.5\bar{x}_1\right)\cos\left(3.2\bar{x}_1\bar{x}_2\right)\cos\left(3.14\bar{x}_2\right)  +\right. \\
  & \left.+0.5\cos\left(2.2\bar{x}_1\right)\cos\left(4.8\bar{x}_1\bar{x}_2\right)\cos\left(3.5\bar{x}_2\right)\right).\nonumber
 \end{align}
 \item \hyperref[TestFunctions:section:HML_TestFunction_MultiplicativePotential]{\textbf{HML\_TestFunction\_MultiplicativePotential}} --- мультипликативная потенциальная функция:
 \begin{equation}
 f\left( \bar{x}\right) = -z\left( \bar{x}_1\right)\cdot z\left( \bar{x}_2\right), \text{ где}
 \end{equation}
 \begin{equation*}
 z\left( v\right)= -\dfrac{1}{\left( v-1\right)^2+0.2 }-\dfrac{1}{2\left( v-2\right)^2+0.15}-\dfrac{1}{3\left( v-3\right)^2+0.3}.
 \end{equation*}
 \item \hyperref[TestFunctions:section:HML_TestFunction_Rana]{\textbf{HML\_TestFunction\_Rana}} --- функция Rana:
 \begin{align}
 f\left( \bar{x}\right) = & \bar{x}_1\sin\left( \sqrt{\left| \bar{x}_2+1-\bar{x}_1\right| }\right) \cos\left( \sqrt{\left| \bar{x}_2+1+\bar{x}_1\right| }\right)+  \\&+ (\bar{x}_2+1)\cos\left( \sqrt{\left| \bar{x}_2+1-\bar{x}_1\right| }\right) \sin\left( \sqrt{\left| \bar{x}_2+1+x\right| }\right).
 \end{align}
 \item \hyperref[TestFunctions:section:HML_TestFunction_RastriginWithChange]{\textbf{HML\_TestFunction\_RastriginWithChange}} --- функция Растригина с изменением коэффициентов:
 \begin{equation}
 f\left( \bar{x}\right) =0.1\bar{x}_1^2+0.1\bar{x}_2^2-4\cos\left( 0.8\bar{x}_1\right) -4\cos\left( 0.8\bar{x}_2\right) +8.
 \end{equation}
 \item \hyperref[TestFunctions:section:HML_TestFunction_RastriginWithTurning]{\textbf{HML\_TestFunction\_RastriginWithTurning}} --- функция Растригина овражная с поворотом осей:
 \begin{align}
 f\left( \bar{x}\right) =&{\left( 0.1 K_{\bar{x}_1}A\left( \bar{x}_1,\bar{x}_2\right) \right) }^2+{\left( 0.1 K_{\bar{x}_2}B\left( \bar{x}_1,\bar{x}_2\right) \right) }^2-
 \\&-4\cos\left( 0.8K_{\bar{x}_1}A\left( \bar{x}_1,\bar{x}_2\right)\right) -4\cos\left( 0.8K_{\bar{x}_2}B\left( \bar{x}_1,\bar{x}_2\right)\right) +8, \text{ где}\nonumber
 \\&A\left( \bar{x}_1,\bar{x}_2\right)= \bar{x}_1\cos\left( \alpha\right) -\bar{x}_2\sin\left( \alpha\right),\nonumber
 \\&B\left( \bar{x}_1,\bar{x}_2\right)= \bar{x}_1\sin\left( \alpha\right) +\bar{x}_2\cos\left( \alpha\right).\nonumber
 \end{align}
 \item \hyperref[TestFunctions:section:HML_TestFunction_ReverseGriewank]{\textbf{HML\_TestFunction\_ReverseGriewank}} --- функция ReverseGriewank:
 \begin{equation}
 f\left( \bar{x}\right) = \dfrac{1}{\dfrac{\bar{x}_1^2+\bar{x}_2^2}{200}-cos\left( \bar{x}_0\right)cos\left( \dfrac{\bar{x}_2}{\sqrt{2}}\right)+2  }, \text{ где}
 \end{equation}
 \item \hyperref[TestFunctions:section:HML_TestFunction_ShekelsFoxholes]{\textbf{HML\_TestFunction\_ShekelsFoxholes}} --- функция "Лисьи норы" Шекеля:
 \begin{equation}
 f\left( \bar{x}\right) = \dfrac{1}{\dfrac{1}{K}+\sum_{j=1}^{25}\dfrac{1}{j+\left( \bar{x}_1-A_{1,j}\right)^6+ \left( \bar{x}_2-A_{2,j}\right)^6}}, \text{ где}
 \end{equation}
 \begin{equation*}
 A =  \bigl(\begin{smallmatrix}
 -32 & -16 & 0 & 16 & 32 & -32 & -16 & 0 & 16 & 32 & -32 & -16 & 0 & 16 & 32 & -32 & -16 & 0 & 16 & 32 & -32 & -16 & 0 & 16 & 32\\
 -32 & -32 & -32 & -32 & -32 & -16 & -16 & -16 & -16 & -16 & 0 & 0 & 0 & 0 & 0 & 16 & 16 & 16 & 16 & 16 & 32 & 32 & 32 & 32 & 32
 \end{smallmatrix}\bigr).
 \end{equation*}
 \item \hyperref[TestFunctions:section:HML_TestFunction_Sombrero]{\textbf{HML\_TestFunction\_Sombrero}} --- функция Сомбреро:
 \begin{equation}
 f\left( \bar{x}\right) =\dfrac{1-{sin\left( \sqrt{\bar{x}_1^2+\bar{x}_2^2}\right)}^2 }{1+0.001\left(\bar{x}_1^2+\bar{x}_2^2 \right) }.
 \end{equation}
 \item \hyperref[TestFunctions:section:HML_TestFunction_Multiextremal]{\textbf{HML\_TestFunction\_Multiextremal}} --- функция Multiextremal:
 \begin{equation}
 f\left( \bar{x}\right) = 0.05\left( x-1\right)^2 + \left( 3-2.9e^{-2.77257x^2}\right)\left( 1-\cos\left(x\left(4-50e^{-2.77257x^2} \right)  \right) \right).
 \end{equation}
 \item \hyperref[TestFunctions:section:HML_TestFunction_Multiextremal2]{\textbf{HML\_TestFunction\_Multiextremal2}} --- функция Multiextremal2:
 \begin{equation}
 f\left( \bar{x}\right) =1-0.5\cos\left( 1.5\left( 10x-0.3\right) \right)\cos\left( 31.4x\right)+0.5\cos\left(\sqrt{5}\cdot10x \right)\cos\left( 35x\right). 
 \end{equation}
 \item \hyperref[TestFunctions:section:HML_TestFunction_Wave]{\textbf{HML\_TestFunction\_Wave}} --- волна:
 \begin{equation}
 f\left( \bar{x}\right) = e^{ -\bar{x}_1^2}+0.01\cos\left( 200\cdot\bar{x}_1\right).
 \end{equation}
 \item \hyperref[TestFunctions:section:HML_TestFunction_SumVector]{\textbf{HML\_TestFunction\_SumVector}} --- сумма всех элементов бинарного вектора:
 \begin{equation}
 f\left( \bar{x}\right) = \sum_{i=1}^{n}\bar{x}_i.
 \end{equation}
 \end{enumerate}