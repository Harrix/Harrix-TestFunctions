\chapter*{Введение}
\addcontentsline{toc}{chapter}{Введение}

В данном документе рассмотрено множество тестовых функций, которые можно использовать для проведения исследований алгоритмов оптимизации. К каждой функции дано подробное описание, график (если это возможно), свойств и параметров, которые позволят единообразно проводить сравнения разных алгоритмов оптимизации во избежания несостыковок с точки зрения разного понимания нахождения ошибки, точности работы алгоритмом.

Данный документ представляет его версию \textbf{1.19} от \today

Последнюю версию документа можно найти по адресу:

\href{https://github.com/Harrix/HarrixTestFunctions}{https://github.com/Harrix/HarrixTestFunctions}

Там же можно найти реализацию тестовых функция в среде Mathcad.

Тестовые функции реализованы на языке C++ в библиотеке  \textbf{HarrixMathLibrary} в разделе <<Тестовые функции для оптимизации>>, которую можно найти по адресу:

\href{https://github.com/Harrix/HarrixMathLibrary} {https://github.com/Harrix/HarrixMathLibrary}.

Все библиографические материалы, которые используются в документе, приведены в виде скриншотов, скринов, документов в папке \textbf{\_Biblio} на \href{https://github.com/Harrix/HarrixTestFunctions}{https://github.com/Harrix/HarrixTestFunctions}.

С автором можно связаться по адресу \href{mailto:sergienkoanton@mail.ru}{sergienkoanton@mail.ru} или  \href{http://vk.com/harrix}{http://vk.com/harrix}.

Сайт автора, где публикуются последние новости: \href{http://blog.harrix.org/}{http://blog.harrix.org/}, а проекты располагаются по адресу \href{http://harrix.org/}{http://harrix.org/}.


\clearpage