\section {Сумма всех элементов бинарного вектора}
\label{TestFunctions:section:HML_TestFunction_SumVector}
\subsection {Описание функции}

\begin{tabularwide}
\textbf{Идентификатор:} & HML\_TestFunction\_SumVector. \\
\textbf{Наименование:} & Сумма всех элементов бинарного вектора. \\
\textbf{Тип:} & Задача бинарной оптимизации. \\
\end{tabularwide}

\textbf{Формула} (целевая функция):
\begin{equation}
\label{TestFunctions:eq:HML_TestFunction_SumVector}
f\left( \bar{x}\right) = \sum_{i=1}^{n}\bar{x}_i, \text{ где}
\end{equation}
\indent $\bar{x}\in X$, $\bar{x}_j\in \left\lbrace 0; 1 \right\rbrace  $, $j=\overline{1,n}$.

\begin{tabularwide}
\textbf{Обозначение:} &\specialcell{$\bar{x}$ --- бинарный вектор;\\$n$ --- размерность бинарного вектора.}  \\
\textbf{Объем поискового пространства:} & $\mu\left( X\right)=2^n $.   \\
\textbf{Решаемая задача оптимизации:} & $\bar{x}_{max}= \arg \max_{\bar{x}\in X} f\left( \bar{x}\right)$.   \\
\textbf{Точка максимума:} & $\bar{x}_{max}={\left( 1,1,\ldots,1\right)}^\mathrm{T} $, то есть $\left(\bar{x}_{max} \right)_j=1$ ($j=\overline{1,n}$).    \\
\textbf{Максимум функции:} & $f\left(\bar{x}_{max} \right) =n$.   \\
\textbf{Точка минимума:} & $\bar{x}_{min}={\left( 0,0,\ldots,0\right)}^\mathrm{T} $, то есть $\left(\bar{x}_{min} \right)_j=0$ ($j=\overline{1,n}$).    \\
\textbf{Минимум функции:} & $f\left(\bar{x}_{min} \right) =0$.   \\
\end{tabularwide}

\subsection {Основная задача и подзадачи}

\begin{tabularwide}
\textbf{Изменяемый параметр: } & $n$ --- размерность бинарного вектора. \\
\textbf{Значение в основной задаче:} & $n=20$.\\
\textbf{Подзадача №2:} & $n=30$.\\
\textbf{Подзадача №3:} & $n=40$.\\
\textbf{Подзадача №4:} & $n=50$.\\
\textbf{Подзадача №5:} & $n=60$.\\
\textbf{Подзадача №6:} & $n=70$.\\
\textbf{Подзадача №7:} & $n=80$.\\
\textbf{Подзадача №8:} & $n=90$.\\
\textbf{Подзадача №9:} & $n=100$.\\
\textbf{Подзадача №10:} & $n=200$.\\
\end{tabularwide}

\subsection {Нахождение ошибки оптимизации}

Пусть в результате работы алгоритма оптимизации за $N$ запусков мы нашли решения $\bar{x}_{submax}^k$ со значениями целевой функции $f\left( \bar{x}_{submax}^k\right) $ соответственно ($k=\overline{1,N}$). Используем три вида ошибок:

\textbf{Надёжность: }
\begin{equation*}
R = \dfrac{\sum_{k=1}^{N}S\left( \bar{x}_{submax}^k \right) }{N}, \text{ где}
\end{equation*}
\begin{equation*}
S\left( \bar{x}_{submax}^k \right)=\left\lbrace \begin{aligned} 1,& \text{ если } \bar{x}_{submax}^k = \bar{x}_{max} ;   \\ 0,& \text{ иначе}. \end{aligned}\right.
\end{equation*}

\textbf{Ошибка по входным параметрам:}
\begin{equation*}
E_x = \dfrac{\sum_{k=1}^{N} \left( \frac{\sum_{j=1}^{n}\left| \left( \bar{x}_{submax}^k \right)_j-\left( \bar{x}_{max} \right)_j \right| }{n} \right)  }{N}.
\end{equation*}

\textbf{Ошибка по значениям целевой функции: }
\begin{equation*}
E_f = \dfrac{\sum_{k=1}^{N} \left( \frac{\left| f\left( \bar{x}_{submax}^k \right)-f\left( \bar{x}_{max} \right) \right|}{n}\right)   }{N}.
\end{equation*}

\subsection {Свойства задачи}
\begin{tabularwide}
\textbf{Условной или безусловной оптимизации: } & Задача безусловной оптимизации. \\
\textbf{Одномерной или многомерной оптимизации: } & Многомерной: $ n $. \\
\textbf{Функция унимодальная или многоэкстремальная: } & Функция унимодальная. \\
\textbf{Функция стохастическая или нет: } & Функция не стохастическая. \\
\textbf{Особенности: } & Нет. \\
\end{tabularwide}

\subsection {Реализация}

Реализация функции взята из библиотеки HarrixMathLibrary в разделе <<Тестовые функции для оптимизации>>, которую можно найти по адресу \href{https://github.com/Harrix/HarrixMathLibrary} {https://github.com/Harrix/HarrixMathLibrary}.

\begin{lstlisting}[caption=Код функции HML\_TestFunction\_SumVector]
double HML_TestFunction_SumVector(int *x, int VHML_N)
{
/*
Сумма всех элементов бинарного вектора.
Тестовая функция бинарной оптимизации.
Входные параметры:
 x - указатель на исходный массив;
 VHML_N - размер массива x.
Возвращаемое значение:
 Значение тестовой функции в точке x.
*/
double VHML_Result=0;
for (int i=0;i<VHML_N;i++) VHML_Result+=x[i];
return VHML_Result;
}

\end{lstlisting}